\section{Metodología}
Para desarrollar el problema, lo primero que hace el algoritmo es escribir un archivo con n números enteros aleatorios. Posteriormente, se toma este archivo y se leen los distintos números en él hasta llegar el final del archivo. Cada entero se almacena en un vector al que llamaremos vectorprincipal. Luego, se crean 2 vectores de apoyo inicializados en 0, uno contador al que llamaremos contador, y otro con cada número distinto en el texto que llamaremos enterosunicos. Con esto, se recorren todos los enteros del vector principal uno por uno. Se verifica si ese entero es distinto a los almacenados o si es un repetido al compararlo con todos los números del vector enterosunicos, que se recorre desde el inicio hasta la última posición con un número único (esta posición se almacena en la variable (cantidadenterosunicosactuales). 


De esta forma si hay un número repetido se hacen menos comparaciones que con uno nuevo. Se aumenta el contador correspondiente al número en el vector contador y si es un número nuevo se añade al vector enterosunicos.Esto se evidencia en el algoritmo anterior.


Este algoritmo se corre para los 3 posibles casos: mejor caso, caso promedio y peor caso. Para el mejor caso, se crea un archivo con n cantidad de 0; para el caso promedio el archivo tiene n números aleatorios que van desde 0 hasta n; y para el peor caso el archivo tiene n números distintos. Para determinar el comportamiento de la gráfica tiempo vs n para cada caso, se ejecuta el proceso anterior contando cada condicional (Si) que utiliza. Este procedimiento se realiza 300 veces, y se halla un promedio del tiempo de ejecución y complejidad (uso de condicionales). Para el mejor caso se realiza variando el valor de n desde 32 hasta 2000000 aumentando n al doble cada vez. En cambio, para el caso promedio y el peor caso los valores de n van desde 32 hasta 5000, aumentando n en factor de 3/2. Esta diferencia se debe a que un mayor valor de n produciría un tiempo de ejecución muy alto (hace overflow en la variable) y aumentar n al doble en lugar de un factor de 3/2 causaría que la gráfica tuviera muy pocos datos.  


Luego, los datos obtenidos se guardan en un archivo con 3 columnas: n, #comparaciones, tiempo de ejecución. De esta forma se tienen 3 archivos (uno por cada caso), los cuales son usados para graficar los resultados de n vs tiempo con la ayuda de Python y MatLab. Por último, se hace un ajuste de la gráfica para determinar su comportamiento y sacar las conclusiones del experimento. 