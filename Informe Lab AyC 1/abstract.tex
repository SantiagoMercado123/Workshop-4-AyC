\begin{abstract}
En clase hemos analizado las diferencias en tiempo y efectividad del mejor, el peor y el promedio de los casos para diferentes algoritmos. El objetivo de este laboratorio es analizar estos 3 casos en un algoritmo que ya teníamos implementado, este algoritmo crea un archivo con N enteros aleatorios, luego todas esas matrices se transfieren a una matriz. El programa lee cada elemento de la matriz y almacena los números únicos en una matriz diferente. También cuenta la cantidad de veces que aparece cada número único e imprime la información al usuario. Este proceso se realiza al menos 200 veces para cada archivo con n enteros para encontrar el tiempo de ejecución promedio para el programa dado. También fuerza el mejor de los casos (todos los N enteros son 0) para analizarlo y compararlo con el promedio y el peor de los casos (N números diferentes). 
\end{abstract}