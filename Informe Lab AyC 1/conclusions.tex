\section{Conclusion}
En conclusión, podemos decir que todos los objetivos de este laboratorio se cumplieron de manera exitosa al poder observar y analizar el comportamiento de los datos (input size, comparaciones y tiempo de ejecución) para los 3 diferentes casos. Como se logra evidenciar en los resultados de todos los casos, a mayor es el input size (n) mayor es el tiempo de ejecución y también es mayor el número de comparaciones. Sin embargo, el comportamiento de las 3 variables varía según el caso. Para el Mejor caso, podemos concluir que con el aumento de los elementos se aprecia un comportamiento lineal, demostrando que el número de comparaciones es directamente proporcional al tiempo de ejecución del algoritmo. Por otro lado, para el caso promedio y el peor caso a pesar de que las gráficas parecen que tuvieran un comportamiento lineal, esto es simplemente debido al ajuste de la gráfica con la ayuda de Python. El comportamiento que tienen en realidad es cuadrático demostrando que también así la proporcionalidad entre el número de comparaciones y el tiempo de ejecución, sin embargo, esta proporcionalidad se muestra a mayor escala ya que a medida que aumenta el número de comparaciones, el tiempo de ejecución también aumenta, pero a mayor escala. Por otro lado, durante este laboratorio no se nos presentaron muchos inconvenientes específicos, simplemente fue necesario aprender el algoritmo para crear, editar y leer archivos .TXT para posteriormente utilizar el algoritmo otorgado por el profesor en Python para graficar nuestros resultados y lograr nuestros objetivos. Finalmente se lograron todos los resultados esperados y que se habían mencionado en clase, concluyendo así el laboratorio satisfactoriamente.